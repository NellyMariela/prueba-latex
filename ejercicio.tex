\documentclass{article}

\usepackage[utf8]{inputenc}
\usepackage[spanish]{babel}
\usepackage{amsmath}

\begin{document}

1. Escribe la fórmula de la solución a la ecuación en diferencias lineal homogénea de primer orden con condición inicial:$$x_{n+1}=ax_n, x_0=C$$

Solución:

\begin{align*}
  x_1&=aC\\
  x_2&=a^2C\\
  x_3&=a^3C
\end{align*}

Nos podemos dar cuenta que la $a$ se eleva de acuerdo al subíndice de $x$

$x_{1}=aC, x_{2}=a^{2}C, x_{3}=a^{3}C, ..., x_{n}=a^{n}C$

Demostración de que  la fórmula $x_n=a^nC$ es válida.

Por inducción en $n$ sabemos que $x_0=C.$

La fórmula dice que $x_0=a^0C$ entonces $x_0=C.$


Supongamos válida la fórmula para n, i.e.

\begin{align*}
  x_{n+1}&=ax_n\\
  x_{n+1}&=a(a^nC)\\
  x_{n+1}&=a^{n+1}C
\end{align*}

2. Resuelve la ecuación en diferencias: $$x_{n+1}-x_n=e^n, x_0=2$$

Podemos observar que: $x_{n+1}=e^n+x_n$

Entonces:

\begin{align*}
  x_1&=e^0+x_0\\
x_1&=e^0+2\\
x_2&=e^1+x_1\\
  x_2&=e^1+e^0+2
\end{align*}

La fórmula para la ecuación es:
$$x_{n+1}=2+\sum_{i=0}^{n}e^i$$

La demostración de la fórmula se deja al lector.




\end{document}











